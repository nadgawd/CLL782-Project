\documentclass[11pt,a4paper]{article}
\usepackage[utf8]{inputenc}
\usepackage{amsmath, amssymb, amsthm}
\usepackage{geometry}
\usepackage{booktabs}
\usepackage{graphicx}
\usepackage{hyperref}
\usepackage{enumitem}
\usepackage[table,xcdraw]{xcolor}
\usepackage{fancyhdr}

% Page Setup
\geometry{margin=1in}
\pagestyle{fancy}
\fancyhf{}
\lhead{CLL782: Process Optimization}
\rhead{Module 3.1: Environmental Load Modeling}
\cfoot{\thepage}

% Title Info
\title{\textbf{Process Optimization Project Report}\\
\large Module 3.1: Quantifying Environmental Load for Sustainable "Rendezvous"}
\author{Your Name \\ Entry Number: 202XXXXX \\ Department of Chemical Engineering, IIT Delhi}
\date{\today}

\begin{document}

\maketitle

% Declaration of Tool Usage
\section*{Declaration of Tool Usage}
I declare that in completing this assignment:
\begin{itemize}
    \item I used an LLM-based tool (Gemini) for assistance in:
    \begin{itemize}
        \item Brainstorming model formulations and exploring nonlinear crowding effects.
        \item Formatting the mathematical expressions and generating LaTeX code.
        \item Verifying the symbolic differentiation of the proposed objective function.
    \end{itemize}
    \item I understand the submitted solution fully.
    \item I can explain and justify every part of my code and reasoning.
    \item I have verified all results independently.
\end{itemize}

\vspace{0.5cm}
\hrule
\vspace{0.5cm}

\tableofcontents
\newpage

\section{Introduction}
The "Rendezvous" cultural festival at IIT Delhi presents a significant logistical and environmental challenge. As part of the Sustainability Task Force, our objective in Module 3.1 is to develop a robust mathematical model to quantify the total environmental load ($E$) of the festival. This model balances exploration of complex system dynamics (crowding, economies of scale) with the exploitation of rigorous optimization techniques to identify trade-offs and guide sustainable decision-making.

\section{Nomenclature}
The variables and parameters used in the mathematical model are defined in Table \ref{tab:nomenclature}.

\begin{table}[h]
\centering
\caption{Nomenclature Table}
\label{tab:nomenclature}
\begin{tabular}{@{}llcc@{}}
\toprule
\textbf{Symbol} & \textbf{Description} & \textbf{Units} & \textbf{Type} \\ \midrule
$N$ & Number of attendees & persons & Decision Var \\
$S$ & Number of food stalls/vendors & count & Decision Var \\
$A$ & Total event activities & activity-hours & Decision Var \\
$E$ & Total environmental load & kg CO$_2$-eq & Objective Fn \\
$\alpha_1$ & Base per-capita impact coefficient & kg CO$_2$-eq/person & Parameter \\
$\alpha_2$ & Base per-stall impact coefficient & kg CO$_2$-eq/stall & Parameter \\
$\alpha_3$ & Base per-activity impact coefficient & kg CO$_2$-eq/activity-hr & Parameter \\
$\gamma_{NS}$ & Congestion penalty coefficient & kg CO$_2$-eq $\cdot$ stall/person$^2$ & Parameter \\
$\beta_N$ & Crowding nonlinearity coefficient & kg CO$_2$-eq/person$^{1.3}$ & Parameter \\
$\beta_S$ & Stall scaling nonlinearity coefficient & kg CO$_2$-eq/stall$^{1.2}$ & Parameter \\
$\beta_A$ & Activity scaling nonlinearity coefficient & kg CO$_2$-eq/activity-hr$^{0.8}$ & Parameter \\
$\epsilon$ & Grid emission factor (India) & kg CO$_2$/kWh & Parameter \\
$\tau$ & Average event duration per attendee & hours & Parameter \\
$P_{light}$ & Lighting power per attendee & kW/person & Sub-parameter \\
$P_{stall}$ & Power per food stall & kW/stall & Sub-parameter \\
$P_{stage}$ & Power per activity stage & kW/stage & Sub-parameter \\
$\phi_{waste}$ & Base waste generation per person & kg/person/day & Sub-parameter \\
$k_{crowd}$ & Crowding waste amplification factor & dimensionless & Sub-parameter \\
$\xi$ & Sustainability Efficiency Metric & kg CO$_2$-eq/(person$\cdot$hr) & Derived Var \\
\bottomrule
\end{tabular}
\end{table}

\section{Assumptions and Justifications}
To develop a tractable yet expressive model, the following assumptions are made based on Life Cycle Assessment (LCA) principles and chemical engineering process economics (Letterman, 1980):

\begin{enumerate}
    \item \textbf{A1: Nonlinear Crowding Effects (Diseconomies of Scale)}. 
    \textit{Justification}: Environmental impact does not scale linearly with attendee density. As crowd density increases, waste collection efficiency drops, littering increases (overflowing bins), and resource competition leads to higher per-capita waste generation. We model this with a superlinear term $N^{\beta_1}$ where $\beta_1 > 1$.
    
    \item \textbf{A2: Economie of Scale in Infrastructure}.
    \textit{Justification}: High-fixed-cost infrastructure (generators, stages) becomes more efficient per unit of activity as scale increases. However, the total load still grows. For activities $A$, we use a sublinear power term $A^{0.8}$ for base energy load, reflecting standard equipment scaling laws (0.6-factor rule).
    
    \item \textbf{A3: Congestion-Induced Waste Leakage}.
    \textit{Justification}: When the ratio of attendees to service points (stalls $S$) is high ($N/S \gg \text{capacity}$), "leakage" occurs in the form of unsegregated waste and littering. This is modeled as a penalty term proportional to $N^2/S$.
    
    \item \textbf{A4: Sub-component Decomposition}.
    \textit{Justification}: The total load $E$ is the sum of Energy ($E_{energy}$), Waste ($W_{total}$), and Emissions ($E_{emissions}$), adhering to ISO 14040 LCA framework boundaries (Cradle-to-Grave for site operations).
\end{enumerate}

\section{Mathematical Model Formulation}

\subsection{Objective Function Construction}
We propose a multi-variable objective function $E(N, S, A)$ comprising three components: Linear Base Load ($E_{base}$), Nonlinear Scale Effects ($E_{scale}$), and Interaction/Congestion Penalties ($E_{inter}$).

\begin{equation}
    E(N, S, A) = E_{base} + E_{scale} + E_{inter}
\end{equation}

Substituting the terms:
\begin{equation} \label{eq:full_model}
    E(N, S, A) = (\alpha_1 N + \alpha_2 S + \alpha_3 A) + (\beta_N N^{1.3} + \beta_S S^{1.2} + \beta_A A^{0.8}) + \left( \gamma_{NS} \frac{N^2}{S} \right)
\end{equation}

Where:
\begin{itemize}
    \item $\alpha_1 N$: Direct resource consumption per attendee.
    \item $N^{1.3}$: The superlinear crowding term.
    \item $\gamma_{NS} \frac{N^2}{S}$: The \textbf{Congestion Penalty}. As $S \to 0$ for fixed $N$, environmental load explodes due to waste management failure. As $S \to \infty$, this term vanishes (perfect service), but the direct cost $\alpha_2 S$ increases. This creates a trade-off.
\end{itemize}

\subsection{Sub-Component Integration}
The aggregate function $E$ is derived from specific engineering estimates:

\begin{align}
    E_{energy} &= \epsilon \cdot \tau \cdot (P_{light} N + P_{stall} S + P_{stage} A) \\
    W_{total} &= \phi_{waste} \cdot N \cdot (1 + k_{crowd} \frac{N}{S}) \label{eq:waste_model} \\
    CO2e &= E_{transp} + E_{gen} + E_{mat}
\end{align}

Equation (\ref{eq:waste_model}) explicitly shows how waste generation rate depends on the crowding ratio $N/S$.

\subsection{Parameter Estimation for Rendezvous, IIT Delhi}
To ground the model in a realistic scenario, we estimate parameter values for the ``Rendezvous'' cultural festival at IIT Delhi, held on the institute's 325-acre (131 ha) green campus. Rendezvous attracts approximately 160,000 attendees over 4 days ($\sim$40,000 per day), features $\sim$100 food/vendor stalls and 15--20 parallel activity stages \cite{bettencourt2007, cea2024, cpcb2021, agf2022}. The estimated values are presented in Table~\ref{tab:params}.

\begin{table}[h!]
\centering
\caption{Estimated Parameter Values for Rendezvous, IIT Delhi}
\label{tab:params}
\small
\begin{tabular}{@{}llll@{}}
\toprule
\textbf{Parameter} & \textbf{Value} & \textbf{Justification} & \textbf{Source} \\ \midrule
\multicolumn{4}{l}{\textit{Base Impact Coefficients (Linear Terms)}} \\
$\alpha_1$ & 2.5 kg CO$_2$-eq/person & 35\% of total 7 kg/person/day on-site & \cite{agf2022} \\
$\alpha_2$ & 18 kg CO$_2$-eq/stall & Energy (60 kWh $\times$ 0.727) + materials & \cite{cea2024} \\
$\alpha_3$ & 12 kg CO$_2$-eq/act-hr & 25 kW stage, adj.\ for 50\% renewables & \cite{cea2024} \\
\midrule
\multicolumn{4}{l}{\textit{Nonlinear Scaling Coefficients}} \\
$\beta_N$ & 0.002 & Urban superlinear scaling ($\sim$1.15--1.3) & \cite{bettencourt2007} \\
$\beta_S$ & 0.5 & Supply-chain logistics fragmentation & \cite{letterman1980} \\
$\beta_A$ & 5.0 & Six-tenths rule, exponent 0.8 & \cite{williams1947, chilton1950} \\
\midrule
\multicolumn{4}{l}{\textit{Congestion Penalty}} \\
$\gamma_{NS}$ & 0.0005 & $S^* \approx 210$ stalls at $N$=40,000 & \cite{agf2022} \\
\midrule
\multicolumn{4}{l}{\textit{Physical / Engineering Constants}} \\
$\epsilon$ & 0.727 kg CO$_2$/kWh & Indian grid FY 2023--24 & \cite{cea2024} \\
$\tau$ & 6 hours & Avg.\ student attendance at campus fest & \cite{prakash2025} \\
\midrule
\multicolumn{4}{l}{\textit{Sub-Component Parameters}} \\
$P_{light}$ & 0.05 kW/person & LED area lighting, 5 W/m$^2$ & \cite{cea2024} \\
$P_{stall}$ & 7 kW/stall & Midpoint of 5--12 kW food truck range & \cite{cea2024} \\
$P_{stage}$ & 25 kW/stage & Sound (15) + LED (5) + AV (5) kW & \cite{cea2024} \\
$\phi_{waste}$ & 0.65 kg/person/day & Delhi per-capita MSW & \cite{cpcb2021} \\
$k_{crowd}$ & 0.002 & Waste $\times$1.75 at $N/S \approx 400$ & \cite{agf2022} \\
\bottomrule
\end{tabular}
\end{table}

\textbf{Key Observation}: With these values, the congestion term $\gamma_{NS} N^2/S = 0.0005 \times 40000^2 / 100 = 8{,}000$ kg CO$_2$-eq dominates the environmental load at full capacity, underscoring the critical importance of sufficient stall infrastructure. The optimal stall count $S^* \approx N\sqrt{\gamma_{NS}/\alpha_2} = 40000\sqrt{0.0005/18} \approx 210$ stalls.

\subsection{Region of Interest Analysis}
Based on the official layout plan (Region.pdf) and campus map, the festival's primary ``Region of Interest'' (ROI) spans a central band covering approximately \textbf{93 acres} (37.6 hectares), representing about 29\% of the total 320-acre IIT Delhi campus. This region encompasses the high-activity zones in the West (Nalanda/SAC), Center (Sports Complex), and East (Academic Core to Main Gate).

For the purpose of spatial modeling in Module 3.2, this ROI is divided into a fine grid of \textbf{155 cells}, each approximately \textbf{0.6 acres} ($50\text{m} \times 50\text{m}$) in size, to allow for high-resolution crowd density analysis. The breakdown of sub-zones is detailed in Table \ref{tab:roi_refined}.

\begin{table}[h]
\centering
\caption{Refined Region of Interest Sub-Zones for Rendezvous}
\label{tab:roi_refined}
\begin{tabular}{@{}llcc@{}}
\toprule
\textbf{Zone} & \textbf{Description} & \textbf{Area (acres)} & \textbf{Type} \\ \midrule
West & SAC, OAT, Nalanda, Parking & 22 & Event Venue \\
Center & Sports Complex (Indoor/Outdoor) & 28 & Open Ground \\
East & Academic Core, LHC to Main Gate & 35 & Academic \\
Circulation & Connecting Pathways & 8 & Roads \\
\midrule
  & \textbf{Total ROI} & \textbf{93} & --- \\
\bottomrule
\end{tabular}
\end{table}

\section{Optimization Analysis}

\subsection{Unconstrained Optimization of $E$}
We seek to minimize $E(N, S, A)$ over the domain $N, S, A \ge 0$.

\textbf{Gradient Check}:
\begin{align}
    \frac{\partial E}{\partial N} &= \alpha_1 + 1.3 \beta_N N^{0.3} + \frac{2 \gamma_{NS} N}{S} > 0 \quad (\text{always positive for } N,S > 0) \\
    \frac{\partial E}{\partial A} &= \alpha_3 + 0.8 \beta_A A^{-0.2} > 0
\end{align}
Since the gradient components for productivity variables ($N, A$) are strictly positive, the unconstrained global minimum lies at the trivial solution:
\begin{equation}
    (N^*, S^*, A^*) = (0, 0, 0)
\end{equation}
\textbf{Insight}: Minimizing total environmental footprint to zero implies having no festival. This formulation, while chemically accurate, does not capture the \textit{value} of the event.

\subsection{Sub-Optimization: Optimal Stall Count}
However, for a \textbf{fixed number of attendees} $N$, there exists an optimal infrastructure level $S^*$. Differentiating $E$ with respect to $S$:

\begin{equation}
    \frac{\partial E}{\partial S} = \alpha_2 + 1.2 \beta_S S^{0.2} - \gamma_{NS} \frac{N^2}{S^2}
\end{equation}

Setting $\frac{\partial E}{\partial S} = 0$:
\begin{equation}
    \alpha_2 + 1.2 \beta_S S^{0.2} = \gamma_{NS} \frac{N^2}{S^2}
\end{equation}

The LHS represents the \textit{Marginal Environmental Cost} of adding a stall (embodied carbon, power). The RHS represents the \textit{Marginal Environmental Benefit} of reducing congestion (less litter). 
The unique intersection $S^*(N)$ defines the optimal resource allocation for any crowd size. 
Approximating for small nonlinearities ($\beta_S \approx 0$):
\begin{equation}
    S^* \approx N \sqrt{\frac{\gamma_{NS}}{\alpha_2}}
\end{equation}
This square-root law suggests that infrastructure should scale linearly with crowd size to maintain constant congestion levels $(N/S)$, but considering base costs, an optimal ratio emerges.

\subsection{Sustainability Efficiency (Part e)}
To address the trivial minimum problem, we refine the objective to minimize \textbf{Impact Intensity} or maximize \textbf{Sustainability Efficiency} ($\xi$):

\begin{equation}
    \text{minimize } \xi(N, S, A) = \frac{E(N, S, A)}{N \cdot \tau}
\end{equation}

This function represents the environmental cost per "attendee-hour".
\begin{align}
    \xi &= \frac{\alpha_1}{\tau} + \frac{\alpha_2 S}{N\tau} + \frac{\alpha_3 A}{N\tau} + \frac{\gamma_{NS} N}{S\tau} + \dots
\end{align}
Minimizing $\xi$ with respect to $N$ now involves trade-offs between fixed environmental costs (stalls, stages) which benefit from economies of scale (larger $N$), and crowding costs ($\gamma_{NS} N/S$) which penalize scale. This formulation yields a non-trivial measure of the "greenest" event scale.

\section{Preliminary Insights and Discussion}
\begin{itemize}
    \item \textbf{Crowding is Environmentally Expensive}: The term $\gamma_{NS} N^2/S$ is the dominant driver of inefficiency at large scales. Investing in sufficient infrastructure ($S$) is not just operationally necessary but environmentally critical to prevent waste leakage.
    \item \textbf{Economies of Scale}: The sublinear energy term for activities ($A^{0.8}$) suggests that consolidating entertainment into fewer, high-quality events is greener than many small, fragmented setups.
    \item \textbf{Critical Ratio}: The analysis reveals a critical Attendee-to-Stall ratio ($N/S)_{crit} = \sqrt{\alpha_2/\gamma_{NS}}$. Operating above this ratio causes environmental load to spike due to congestion.
\end{itemize}

\begin{thebibliography}{99}

\bibitem{bettencourt2007}
Bettencourt, L.M.A., Lobo, J., Helbing, D., K\"uhnert, C. \& West, G.B. (2007).
Growth, innovation, scaling, and the pace of life in cities.
\textit{Proceedings of the National Academy of Sciences}, 104(17), 7301--7306.

\bibitem{cea2024}
Central Electricity Authority (2024).
CO$_2$ Baseline Database for the Indian Power Sector, Version 20.0.
Ministry of Power, Government of India.

\bibitem{cpcb2021}
Central Pollution Control Board (2021).
Annual Report on Municipal Solid Waste Management.
MoEF\&CC, Government of India. Delhi per-capita: 0.65 kg/day.

\bibitem{chilton1950}
Chilton, C.H. (1950).
Six-Tenths Factor Applies to Complete Plant Costs.
\textit{Chemical Engineering}, 57(4), 112--114.

\bibitem{edgar2001}
Edgar, T.F., Himmelblau, D.M. \& Lasdon, L.S. (2001).
\textit{Optimization of Chemical Processes}, 2nd ed. McGraw-Hill.

\bibitem{iso14040}
ISO 14040:2006.
Environmental management --- Life cycle assessment --- Principles and framework.

\bibitem{letterman1980}
Letterman, R.D. (1980).
Economic analysis of granular-bed filtration.
\textit{Journal of the Environmental Engineering Division}, 106, 279--291.

\bibitem{prakash2025}
Prakash, O. (2025).
\textit{Course Lecture Notes: Project Details}, CLL782/CHL7204, IIT Delhi.

\bibitem{agf2022}
Sherburn, M. et al. (2022).
``The Show Must Go On'' --- 2022 Event Industry GHG Emissions Report.
A Greener Future (AGF).

\bibitem{williams1947}
Williams, R. Jr. (1947).
Six-tenths factor aids in approximating costs.
\textit{Chemical Engineering}, 54(12), 124--125.

\end{thebibliography}

\end{document}
